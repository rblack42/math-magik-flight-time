\section{Python Packages}

We will be using two nice Python packages to support this article: |SYMPY\ and \PINT. These
packages are easily installed using the \PY\ {\bf pip} tool. Here is the line I
used to install them on my system:

\begin{Verbatim}[frame=single]
$ pip install sympy, pint
\end{Verbatim}

\subsection{SymPy}

We will be working through a bit of math in this article. I will show the
equations in their ``normal'' form but I will be using a nice {\it Python} package
called {\it SymPy} to generate the actual results. {\it SymPy} is a symbolic
math engine, it manipulates {\it symbols} the same way you did back in your
algebra class in high school. The really nice thing about {\it SymPy} is that
it knows a lot about math, and it does not make mistakes (well, unless you do
when you set things up!)

As a simple example of what \SYMPY\ can do, let's show a bit of that algebra.
Suppose you were asked to expand $(x + y)^3$. Here is the \PY\ code that uses
\SYMPY\ to figure this out for you:

\setpygmentsfv{frame=single, xrightmargin=1cm}
\begin{pyblock}
import sympy
sympy.var("x, y")

sol = sympy.expand((x + y)**3)

print(r"\begin{align*}")
print(sympy.latex(sol))
print(r"\end{align*}")
\end{pyblock}

Those {\bf print} statements are strictly for use in this document, so things
display nicely in the final {\it PDF file}. The result is this:

\begin{pycode}
from sympy import *
var("x, y")

sol = expand((x + y)**3)

print(r"\begin{align*}")
print(latex(sol))
print(r"\end{align*}")
\end{pycode}

I bet you wish you had access to this tool back in your school days!

\subsection{\PINT}

In engineering, we absolutely must pay attention to the units we use for
physical properties. In the U.S. we commonly use Imperial units (inches or
feet, and ounces or pounds). Elsewhere it is more common to use Metric units
(centimeters, or meters, and grams or kilograms). Indoor modelers seem to mix
the two systems, using inches for lengths, and grams for weight. Failure to pay
attention to all of this can lead to embarrassing failures, as NASA found out in
one of their Mars expeditions when some subcontractors were using Metric units
and everyone else was using Imperial units!

Fortunately, another \PY| package can make tracking units and converting them
as needed easy. The |PINT| package has a lot of power we can use here:

\begin{pyblock}
from pint import UnitRegistry
ureg = UnitRegistry()

rho = 0.002308 * (ureg.slug / ureg.ft**3)
print('{:~P}'.format(rho))
\end{pyblock}

Once again, we see some trickery to get the output to look nice in a document generated using \LaTeX.

Here is the result:

\begin{pycode}
from pint import UnitRegistry
ureg = UnitRegistry()

rho = 0.002308 * (ureg.slug / ureg.ft**3)
print('{:~P}'.format(rho))

\end{pycode}

Who came up with ``slug'' as a unit anyway? Let's convert that to the Metric system:

\begin{pyblock}
rho_m = rho.to_base_units()
print('{:~P}'.format(rho_m))
\end{pyblock}

\PINT\ uses the Metric system internally, so we just asked it to show those
``base'' units:

\begin{pycode}
rho_m = rho.to_base_units()
print('{:~P}'.format(rho_m))
\end{pycode}

This is really nice! No more digging out a calculator and googling for unit
conversion factors!

Armed with these tools, let's take a look at the atmosphere we will be flying
through next.
